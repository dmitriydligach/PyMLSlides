\documentclass{beamer}
\usepackage{latexsym}
\usepackage{graphicx}
\usepackage{hyperref}
\usetheme{Warsaw}

\title{ML Background}
\subtitle{Maximum likelihood etc.}

\begin{document}
\maketitle

\begin{frame}
  \frametitle{Coin Tossing}
  \begin{itemize}
  \item Given a coin, find out $P(heads)$
  \item I.e. the probability that if you flip it, it lands as `heads' \pause
  \item Flip it a few times: $H$ $H$ $T$
  \item $P(heads)=2/3$, no need for Comp 379
  \item Hmm... is this rigorous?
  \end{itemize}
\end{frame}

\begin{frame}
  \frametitle{Bernoulli distribution}
  \begin{itemize}
  \item Single binary random variable $x\in\{0,1\}$
  \item E.g. $x=1$ represents `heads' and $x=0$ represents `tails'
  \item Probability of $x=1$ denoted by the parameter $\mu$
  \item So, $p(x=1|\mu) = \mu$ and $p(x=0|\mu) = 1 - \mu$
  \item The probability distribution over $x$ can be written
  \end{itemize}
  \centering
  $Bern(x|\mu) = \mu^x(1-\mu)^{1-x}$
\end{frame}

\begin{frame}
  \frametitle{Coin tossing model}
  \begin{itemize}
  \item Assume coin flips are independent and identically distributed
  \item All are separate samples from the Bernoulli distribution (i.i.d.)
  \item Given data $\mathcal{D} = \{x_1,\ldots,x_N\}$
  \item Where heads: $x_i=1$ and tails: $x_i=0$
  \item The \textbf{likelihood} of the data is: \[p(\mathcal{D}|\mu) = \prod_{n=1}^{N} p(x_n|\mu) = \prod_{n=1}^{N} \mu^{x_n} (1-\mu)^{1-x_n} \]
  \end{itemize}
\end{frame}

\begin{frame}
  \frametitle{Maximum Likelihood Estimation}
  \begin{itemize}
  \item Given $\mathcal{D}$ with $H$ heads and $T$ tails
  \item What should $\mu$ be?
  \item Maximum Likelihood Estimation (MLE)
  \item Choose $\mu$ which maximizes the likelihood of the data
    \[ \mu_{ML} = \arg \max_{\mu} p(\mathcal{D}|\mu) \]
  \item Since $\ln(\cdot)$ is monotonically increasing:
     \[ \mu_{ML} = \arg \max_{\mu} \ln p(\mathcal{D}|\mu) \]
  \end{itemize}
  \tiny
  \textbf{NOTE:} A monotonically increasing function is one that increases as $x$ does for all real $x$
\end{frame}

\begin{frame}
  \frametitle{Maximum Likelihood Estimation}
  \begin{itemize}
  \item Likelihood
    \[ p(\mathcal{D}|\mu) = \prod_{n=1}^{N} \mu^{x_n} (1-\mu)^{1-x_n} \]
  \item Log-likelihood
    \[ \ln p(\mathcal{D}|\mu) = \sum_{n=1}^{N} x_n \ln \mu + (1-x_n) \ln (1-\mu) \]
  \item Take the derivative and set to 0 \pause
    \[ \frac{d}{d \mu } \ln p(\mathcal{D}|\mu) = \sum_{n=1}^{N} x_n \frac{1}{\mu} - (1-x_n) \frac{1}{1-\mu}  = \frac{1}{\mu} H - \frac{1}{1-\mu} T \]
    \[ \mu = \frac{H}{T + H} \]
  \end{itemize}
\end{frame}

\begin{frame}
  \frametitle{Visualize likelihood function}
  \begin{itemize}
  % Plot[p^4(1-p)^6,{p,0,1}]
  \item Plot[p\textasciicircum4(1-p)\textasciicircum6,\{p,0,1\}]
  \item \href{https://www.wolframalpha.com/}{Type this into Wolfram Alpha}
  \end{itemize}
\end{frame}

\begin{frame}
\textbf{Acknowledgements:} Slides based on the latex source provided by Oliver Schulte and Greg Mori (Simon Fraser University)
\end{frame}

\end{document}
